\documentclass[.../main.tex]{subfiles}
\graphicspath{{\subfix{.../Figures/}}}
\begin{document}
	
In the previous sections, the ideas of integrals and forms have been developed and explored, and this section shall see their combination. In general, it is known that the input of the integral is to consist of a differential form and a region over which this form is integrable. The simplest case of this is the integration of a $1$-form over an $n$-manifold, often considered as taking the integral of the form over the region embedded in $(n+1)$-space, however this is not the most general. In the same way that the wedge product of two forms of different dimension is more complex than that of two forms of the same dimension, so too is the integral of an $m$-form over a $p$-manifold when $m<p$. As such it shall be beneficial to first examine the integral of a 1-form over a manifold dimension $k$.

\subsection{The integral of a 1-form over a k-manifold}
From section \ref{subsection: 2.1} it is evident that the integral should preserve orientation, and also that the integral of a form over a region must depend exclusively on the boundaries of that region. We may first consider the regions $V \subset \R{k}$ and $U \subset \R{k}$ (to distinguish between the two the local coordinates in $V$ shall be denoted by $y_i$ and the local coordinates in $U$ shall be denoted by $x_i$), the function $h: V \rightarrow U$, and the differential form $\omega \in \Lambda(U)$. However, because $\omega \in \Lambda(U)$ and $h$ does not take an input in $U$ (and therefore any of its exterior powers), the form may not immediately be integrated over $V$, and there must exist some differential $1$-form defined exclusively from $\omega$ and $h$ which is an element of $\Lambda(U)$.



%Footnotes
\footnotetext{Guillemin and Pollack 163 \cite{guilleminDifferentialForms2010}}

\end{document}
