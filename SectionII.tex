\documentclass[.../main.tex]{subfiles}
\graphicspath{{\subfix{.../Figures/}}}
\begin{document}

In this section, an examination of the integrals of multivariate calculus will be provided, with the aim of identifying a list of properties which are shared between them and which must be therefore shared by the general integral. Further, a series of notational and algebraic conventions will be introduced during this section, the latter of which will be formalized in the subsequent section regarding the exterior algebras of differential forms. The reason for doing this is that, bluntly, these conventions will prove themselves to yield more naturally to the sort of abstraction sought by this paper. A heavier focus on the integrands shall be placed than is typical, and, in general, the notations of vector calculus shall be avoided in favor of more discrete terms which allow for patterns to more readily be noticed.

\subsection{Integrals of one variable} \label{subsection: 2.1}
First, consider the classic definite integral of a function $f$ of 1 variable $x$ on an interval $[a, b] \in \R{2}$,
\[
	\int\limits_{[a, b]} f(x) dx,
\]
which shall, for the purposes of this investigation, not be understood in terms of its limit definition\footnotemark. This integral is the type which is the focus of calculus II, and thus is a good place to start when looking for properties. Let it be noted that the notation used places the limits at the bottom, as this will be more useful later, but it is equivalent to writing $\int^b_a f(x)dx$.

The most immediate thought in terms of properties of the definite interval of a one-variable function is the 'fundamental theorem of calculus\footnotemark, and therefore is a good starting place for generalization:
\begin{theorem}[The fundamental theorem of calculus.]
	For some function $f:\mathbb{R}\rightarrow\R{2}$ which is continuously differentiable on $a, b \in \mathbb{R}$, 
	\[
		\int\limits_{[a, b]}\frac{d}{dx} f(x)dx=f(b)-f(a) 
	\]
	\label{theorem: 2.1}
\end{theorem}
The implications of this theorem are numerous, such as that the result of integrating a function depends solely upon the value of the function at its endpoints, as well as that
\begin{equation}
	\int\limits_{[a, b]}f(x)dx = \int f(b) dx - \int f(a) dx = F(b)-F(a)
	\label{eq: 2.1}
\end{equation}
where $F(x)$ is the family of curves satisfying $F'(x)-f(x)=0$. Further, it immediately follows that 
\begin{equation}
	\int\limits_{[a, a]}f(x)dx = 0.
	\label{eq: 2.2}	
\end{equation} 
It is then likewise evident that the interval over which the integration is being performed is a 1-dimensional space with an orientation, that is, $[a, b]=-[b, a]$, and as such the integral over this space shall naturally preserve this orientation. This is shown by 
\begin{equation}
	\int\limits_{[a, b]} f(x)dx = F(b)-F(a) = -F(a)-F(b) = -\int\limits_{[b, a]}f(x)dx,
	\label{eq: 2.3}	
\end{equation}

% Footnotes
\footnotetext[1]{\raggedright Weisstein, Eric W. "Riemann Integral." From MathWorld--A Wolfram Web Resource. \cite{weissteinRiemannIntegral}}
\footnotetext{\raggedright Note that no proofs shall be provided for any theorems which are standard material up to calculus IV}

\subsection{Integrals of two and three variables} \label{subsection: 2.2}
When studying calculus, integrals of two and three variables are often considered together, as they are both used to describe surfaces in 3-dimensional space, as well as the fact that their intuitions are almost identical. It is because of the similarities of the two integrals that they shall be of much importance to this investigation, as it shall be a great deal easier to recognize patterns in these objects. The fundamental theorem of calculus in \ref{theorem: 2.1} is as essential to the study of 1-integrals as the theorems of Green and Stokes are to the study of 2- and 3- dimensional integrals. As such, much of this section shall concern these two theorems and their implications, and they shall arise once more in the later sections as natural results of the generalized integral.

\paragraph*{Integrals of two dimensions. The theorem of Green in the plane.}

% Footnotes


\subsection{Change of Variables}
Often, an integral is much easier to compute when a change of basis is induced, and as such it would be greatly beneficial to have an integrand which transforms naturally upon a change of basis. A very common example of change of basis is given in many calculus II courses when computing integrals which would be near impossible to solve by hand; this is is, of course, $u$- and trigonometric substitutions. However this is much more than a simple mathematical trick for making calculations easier, and in fact there is a great deal of theory that goes into the transformation properties of the integrand.

A perfect example of this is the trigonometric substitution, which transforms the integral into polar coordinates, in other a transformation from Cartesian coordinates to curvilinear coordinates. This is a difficult transformation, however, because one cannot expect a cartesian basis to transform linearly into a curvilinear basis, and as such the transformation of the integrand is more than a simple substitution. Although it cannot be taken as given that there exists a linear relation between the two bases, it may be taken that there exists a linear relation of their differentials, that is, the $dx$ at the end of the integrand. As such, under the transformation 
\begin{equation}
	\begin{split}
		x &= \cos{t} \\
		y &= \sin{t}
	\end{split}
	\label{eq: 2.4}
\end{equation}
the differential $dx$ should transform according to\footnotemark
\begin{equation}
	dx = \frac{dx}{dt} dt
	\label{eq: 2.5}
\end{equation}
or for this case
\begin{equation}
	dx = -\sin{(t)} dt.
	\label{eq: 2.6}
\end{equation}

The notion of a variable change translates to higher dimensions as well, for instance we shall consider another case of trigonometric substitution, this time into spherical coordinates. The system of spherical coordinates may be described adequately by three variables: the latitude $u^1$, the longitude $u^2$, and the radius $r$. These correspond to the Cartesian system of coordinates according to
\begin{equation}
  \begin{split}
  	x &= \phi_1 (r, u^1, u^2) = r \Co{u^1} \Si{u^2} \\
  	y &= \phi_2 (r, u^1, u^2) = r \Si{u^1} \Si{u^2} \\
  	z &= \phi_3 (r, u^1, u^2) = r \Co{u^2}
  \end{split}
	\label{eq: 2.7}
\end{equation}

Then the Cartesian volume form $dV = dxdydz$ must also transform accordingly when the basis vectors are switched, however this transformation, like \ref{eq: 2.6}, is to be nonlinear and depend heavily on the value $\Si{u^1}$, as is illustrated by \ref{eq: 2.7}. Here it should be clear that these coefficients should be the Jacobian determinant of the transformation, $\mathcal{D}_{\phi_i} (r, u^1, u^2)$, which describes how exactly the volume of an infinitesimal cube scales under this transformation of basis. The Jacobian matrix is defined to be the partial differentials of the new basis vectors with respect to the old\footnotemark, or:

\begin{equation}
	\mathcal{J}_{ij} = \frac{\partial \phi_i}{\partial u^j}
	\label{eq: 2.8}
\end{equation}
so for \ref{eq: 2.7} equation \ref{eq: 2.8} becomes

\begin{equation}
	\mathcal{J}_{ij} = 
	\begin{pmatrix}
		\Si{u^1}\Co{u^2} & -r\Si{u^1}\Si{u^2} & r\Co{u^1}\Co{u^2} \\
		\Si{u^1}\Si{u^2} & r\Co{u^1}\Si{u^2} & r\Si{u^1}\Co{u^2} \\
		\Co{u^2} & 0 & -r\Si{u^2}
	\end{pmatrix}
	\label{eq: 2.9}
\end{equation}
and the determinant becomes

\begin{equation}
	\mathcal{D} = -r\Si{u^1};
	\label{eq: 2.10}
\end{equation}
Thus, the volume form for spherical coordinates is $-r\Si{u^1} drdu^1du^2$.

In general, for some diffeomorphism of two subsets of $\R{l}$, $\phi: U \rightarrow V$, where the bases of $U$ and $V$ are $x_i$ and $y_i$ respectively, the volume form on $V$ is given by $dV = \mathcal{D}_{\phi} (y_i) dy^I$. One last formalism now stands in the way of a general formula for the change of variables of an integrand, namely, how does the argument of the integrand transform, however this is nearly trivial. When given a function $f: U \rightarrow U$ and the function $\phi$ as defined above, the action of $f$ done on $U$ shall be equivalent to the action of $f \circ \phi$ done on $V$.

As such the transformation of an integrand, whose general form is that of 
\begin{equation}
	\omega = \sum\limits_{I}\psi_Idx^I
	\label{eq: 2.11}
\end{equation}
(this notation is introduced formally as a differential form, however for this section they need not be defined as separate from the general integrand), may be defined. This result shall be foundational to the latter sections.

\begin{theorem}[Change of variables in an integrand]
	Consider the integrand $\omega$ defined by \ref{eq: 2.11} on $U \subset \R{n}$ (local coordinates $x_i$) and the diffeomorphism $\phi : U \rightarrow V$ ($V \subset \R{l}$ with local coordinates $y_i$). Then $\omega$ is defined on $V$ according to 
	\[
		\omega = \psi \circ \phi \mathcal{D}_{\phi}(x_i) dy^I
	\]
	\label{theorem: 2.2}
\end{theorem}

% Footnotes
\footnotetext{\raggedright Aris 140 \cite{aris21Contravariant1989}}
\footnotetext[5]{\raggedright Aris 137 \cite{aris12Proper1989}}
\setcounter{footnote}{5}

\end{document}
